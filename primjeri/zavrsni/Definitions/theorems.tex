%----------------------------------------------------------------------------------------
%	NAMES OF TEOREM, DEFINITION ...
%----------------------------------------------------------------------------------------
\gdef\@Theorem{\ifdefstring{\@language}{croatian}{Teorem}{Theorem}}
\gdef\@Definition{\ifdefstring{\@language}{croatian}{Definicija}{Definition}}
\gdef\@Example{\ifdefstring{\@language}{croatian}{Primjer}{Example}}
\gdef\@Exercise{\ifdefstring{\@language}{croatian}{Zadatak}{Exercise}}
\gdef\@Proof{\ifdefstring{\@language}{croatian}{Dokaz}{Proof}}



%----------------------------------------------------------------------------------------
%	PLAIN THEOREM STYLE
%----------------------------------------------------------------------------------------
\ifthenelse{\equal{\@theoremStyle}{plain}}
{
	\newtheoremstyle{plainBox} % Theorem style name
		{0pt} % Space above
		{0pt} % Space below
		{\normalfont} % Body font
		{} % Indent amount
		{} % Theorem head font
		{} % Punctuation after theorem head
		{0.25em} % Space after theorem head
		{\hspace{-15pt}\small\bfseries\thmname{#1}~\thmnumber{\@ifnotempty{#1}{}\@upn{#2}} % Theorem text (e.g. Theorem 2.1)
	\thmnote{\the\thm@notefont\bfseries\color{black}\itshape~#3.\hspace{0.25em}}\newline} 
		
	\newmdenv[
		skipabove=10pt,
		skipbelow=7pt,
		linewidth=0pt,
		leftmargin=5pt,
		rightmargin=5pt,
	]{plainBoxEnv}
	
	
	\theoremstyle{plainBox}
	\newtheorem{theoremeT}{\@Theorem}[chapter]
	\newtheorem{definitionT}{\@Definition}[chapter]
	\newtheorem{exampleT}{\@Example}[chapter]
	\newtheorem{exerciseT}{\@Exercise}[chapter]
	\newtheorem{proofT}{\@Proof}[chapter]

	\newenvironment{theorem}{\begin{plainBoxEnv}\begin{theoremeT}}{\hfill{\color{black}\tiny\ensuremath{\blacksquare}}\end{theoremeT}\end{plainBoxEnv}}
	\newenvironment{definition}{\begin{plainBoxEnv}\begin{definitionT}}{\hfill{\color{black}\tiny\ensuremath{\blacksquare}}\end{definitionT}\end{plainBoxEnv}}
	\newenvironment{example}{\begin{plainBoxEnv}\begin{exampleT}}{\end{exampleT}\end{plainBoxEnv}}
	\newenvironment{exercise}{\begin{plainBoxEnv}\begin{exerciseT}}{\end{exerciseT}\end{plainBoxEnv}}
	\renewenvironment{proof}{\begin{plainBoxEnv}\begin{proofT}}{\hfill{\color{black}\tiny\ensuremath{\blacksquare}}\end{proofT}\end{plainBoxEnv}}
		
}
{}



%----------------------------------------------------------------------------------------
%	ALTER THEOREM STYLE
%----------------------------------------------------------------------------------------
\ifthenelse{\equal{\@theoremStyle}{alter}}
{
	\newtheoremstyle{plainBox} % Theorem style name
		{0pt} % Space above
		{0pt} % Space below
		{\normalfont} % Body font
		{} % Indent amount
		{} % Theorem head font
		{} % Punctuation after theorem head
		{0.25em} % Space after theorem head
		{\small\bfseries\color{black}\thmname{#1}~\thmnumber{\@ifnotempty{#1}{}\@upn{#2}} % Theorem text (e.g. Theorem 2.1)
	\thmnote{\the\thm@notefont\bfseries\color{black}\itshape--- #3 ---\hspace{0.25em}}} 
		
	\newmdenv[
		skipabove=10pt,
		skipbelow=7pt,
		linewidth=0.5pt,
		roundcorner=10pt,
		leftmargin=5pt,
		rightmargin=5pt,
	]{plainBoxEnv}
	
	
	\theoremstyle{plainBox}
	\newtheorem{theoremeT}{\@Theorem}[chapter]
	\newtheorem{definitionT}{\@Definition}[chapter]
	\newtheorem{exampleT}{\@Example}[chapter]
	\newtheorem{exerciseT}{\@Exercise}[chapter]
	\newtheorem{proofT}{\@Proof}[chapter]

	\newenvironment{theorem}{\begin{plainBoxEnv}\begin{theoremeT}}{\end{theoremeT}\end{plainBoxEnv}}
	\newenvironment{definition}{\begin{plainBoxEnv}\begin{definitionT}}{\end{definitionT}\end{plainBoxEnv}}
	\newenvironment{example}{\begin{plainBoxEnv}\begin{exampleT}}{\end{exampleT}\end{plainBoxEnv}}
	\newenvironment{exercise}{\begin{plainBoxEnv}\begin{exerciseT}}{\end{exerciseT}\end{plainBoxEnv}}
	\renewenvironment{proof}{\begin{plainBoxEnv}\begin{proofT}}{\end{proofT}\end{plainBoxEnv}}
	
}
{}




%----------------------------------------------------------------------------------------
%	FANCY THEOREM STYLE
%----------------------------------------------------------------------------------------
\ifthenelse{\equal{\@theoremStyle}{fancy}}
{

	\newtheoremstyle{blueBox} % Theorem style name
		{0pt} % Space above
		{0pt} % Space below
		{\normalfont} % Body font
		{} % Indent amount
		{} % Theorem head font
		{} % Punctuation after theorem head
		{0.25em} % Space after theorem head
		{\small\sffamily\bfseries\color{FSBblue}\thmname{#1}~\thmnumber{\@ifnotempty{#1}{}\@upn{#2}} % Theorem text (e.g. Theorem 2.1)
		\thmnote{\the\thm@notefont\sffamily\bfseries\color{black}\itshape~#3.\hspace{0.25em}}} % Optional theorem note
		

	\newtheoremstyle{greeneBox} % Theorem style name
		{0pt} % Space above
		{0pt} % Space below
		{\normalfont} % Body font
		{} % Indent amount
		{} % Theorem head font
		{} % Punctuation after theorem head
		{0.25em} % Space after theorem head
		{\small\sffamily\bfseries\color{FSBforestgreen}\thmname{#1}~\thmnumber{\@ifnotempty{#1}{}\@upn{#2}} % Theorem text (e.g. Theorem 2.1)
		\thmnote{\the\thm@notefont\sffamily\bfseries\color{black}\itshape~#3.\hspace{0.25em}}} % Optional theorem note	
	
	
	\newtheoremstyle{orangebox} % Theorem style name
		{0pt} % Space above
		{0pt} % Space below
		{\normalfont} % Body font
		{} % Indent amount
		{} % Theorem head font
		{} % Punctuation after theorem head
		{0.25em} % Space after theorem head
		{\small\sffamily\bfseries\color{FSBorange}\thmname{#1}~\thmnumber{\@ifnotempty{#1}{}\@upn{#2}} % Theorem text (e.g. Theorem 2.1)
		\thmnote{\the\thm@notefont\sffamily\bfseries\color{black}\itshape~#3.\hspace{0.25em}}}	
	
	
	\theoremstyle{blueBox}
	\newmdtheoremenv[skipabove=10pt, skipbelow=7pt,
		backgroundcolor=black!5,
		linecolor=FSBblue, linewidth=1pt, roundcorner=10pt,
		leftmargin=10pt, rightmargin=10pt,
		innerleftmargin=10pt, innerrightmargin=10pt,]{theorem}%
		{\@Theorem}[chapter]%
	
	\newmdtheoremenv[skipabove=10pt, skipbelow=7pt,
		backgroundcolor=FSBforestgreen!5,
		linewidth=0pt, roundcorner=10pt,
		leftmargin=0pt, rightmargin=0pt,
		innerleftmargin=10pt, innerrightmargin=10pt, innertopmargin=5pt, innerbottommargin=5pt]{exercise}%
		{\@Exercise}[chapter]%	

	
	\theoremstyle{greeneBox}
	\newmdtheoremenv[skipabove=10pt, skipbelow=7pt,
		backgroundcolor=FSBblue!7,
		linewidth=0pt, roundcorner=10pt,
		leftmargin=0pt, rightmargin=0pt,
		innerleftmargin=7pt, innerrightmargin=7pt, innertopmargin=7pt, innerbottommargin=7pt]{example}%
		{\@Example}[chapter]%
		
		
	\theoremstyle{orangebox}
	\newmdtheoremenv[skipabove=10pt, skipbelow=7pt,
		backgroundcolor=yellow!10!white,
		%rightline=false, leftline=true, topline=true, bottomline=false,
		linecolor=FSBforestgreen, linewidth=1pt, roundcorner=10pt,
		leftmargin=10pt, rightmargin=10pt,
		innerleftmargin=7pt, innerrightmargin=7pt, innertopmargin=7pt, innerbottommargin=7pt]{definition}%
		{\@Definition}[chapter]%

	\newtheorem{proofT}{\@Proof}[chapter]
	\newmdenv[
		skipabove=10pt, skipbelow=7pt,
		rightline=false, leftline=true, topline=false, bottomline=false,
		linecolor=FSBorange, linewidth=2pt
		backgroundcolor=white!10,
		innerleftmargin=5pt, innerrightmargin=5pt, innertopmargin=5pt, innerbottommargin=5pt,
		leftmargin=10pt, rightmargin=10pt]{pBox}
		
	\renewenvironment{proof}{\begin{pBox}\begin{proofT}}{\hfill{\color{FSBorange}\tiny\ensuremath{\blacksquare}}\end{proofT}\end{pBox}}


}
{}






