\chapter{Uvod}
Jedan od najznačajnijih problema kod konstruiranja mehanizama je optimalna sinteza mehanizma.
Tijekom godina razvijeno je nekoliko metoda za sintezu mehanizama \cite{Chiang2000, Hansen2009} ali svaka od njih primjenjiva je samo na određenim tipovima mehanizma. 
Zbog toga odabir odgovarajuće metode optimiranja mehanizma ovisi o samom mehanizmu koji se želi optimirati tj. aplikaciji mehanizma, potrebnoj numeričkoj točnosti te vremenu koje je potrebno da se postigne optimalno riješenje.

Primjena mehanizma utječe na optimizacijski problem, tj. ograničuje ga. U industrijskim aplikacijama ta ograničenja su dužine elemenata, prostor u koji mehanizam mora stati...\\

Možemo razlikovati dvije vrsta optimalne sinteze mehanizma: dimenzionalna i strukturalna sinteza. 
Dimenzionalna sinteza svodi se na određivanje dimenzija linkova mehanizma koji će omogućiti slijeđenje željene trajektorije ili funkcije, dok su mehanizam i veze između linkova poznati.
Strukturalna sinteza teži je problem jer nam nije poznat mehanizam ni veze između linkova te stoga moramo optimirati i topologiju i dimenzije mehanizma.\\

U ovom radu koristiti će se \acrlong{ga} ili \acrshort{ga} \cite{McCall2005, Cabrera2002} za optimiranje mehanizma.
Referenca na primjer tablice~\ref{tbl:primjer}.


\begin{table}[h!]
\centering
\caption{Primjer tablice}
\label{tbl:primjer}
\begin{tabular}{ccc}
stupac 1 & stupac 2 & stupac 3\\
\hline
\hline
nesto & $a+b$ & nesto\\
$a+b$ & nesto & $a+b$\\
\hline
\end{tabular}
\end{table}




\begin{table}[h!]
	\centering
	\caption[Primjer tablice bolji, test citata]{Primjer tablice bolji, test citata \cite{Chiang2000}}
	\label{tbl:primjer2}
	\begin{tabular}{ccc}
		\toprule
		stupac 1 & stupac 2 & stupac 3\\ 
		\midrule
		nesto & $a+b$ & nesto\\
		$a+b$ & nesto & $a+b$\\ 
		\bottomrule
	\end{tabular}
\end{table}
