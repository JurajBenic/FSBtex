\begin{abstract}

	  are categorized as autonomous or remotely controlled aircraft that hold a many favourable features allowing them a wide range of useful applications. In order to successfully complete a flight mission, the aircraft must comply with the required flight performance indices. Aircraft performance depends strongly on the propulsion system and therefore different flight characteristics require different propulsion implementations. In contrast to well–known benefits of   utilisation, in particular multirotor type of  there is one major drawback regarding the on–board available energy. Most common electrically powered aircraft can last in the air for 15 to 30 minutes, while the best ones may last up to 60 minutes.
	
	Therefore, to overcome the aforementioned drawbacks of purely–electric energy source in multirotor aircraft, alternative propulsion systems combining two different energy sources (hybrid power systems) are considered.
	In this work, a methodological approach to the design of a hybrid propulsion unit for multirotor aircraft consisting of an internal combustion engine, electricity generator and electrochemical battery is described.
	
	For this purpose, analysis and modelling of individual components of the propulsion system have been carried out. Physical parameters of dynamic models are identified by means of experimental measurements. Based on the obtained data, a detailed model of a hybrid electrical propulsion configuration was proposed and suitable control strategies for the hybrid power system have been designed. The proposed control system design methodology has been verified by exhaustive computer simulations and experimentally by using different experimental setups. 
\end{abstract}